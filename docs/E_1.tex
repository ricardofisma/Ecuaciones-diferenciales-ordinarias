\PassOptionsToPackage{unicode=true}{hyperref} % options for packages loaded elsewhere
\PassOptionsToPackage{hyphens}{url}
\PassOptionsToPackage{dvipsnames,svgnames*,x11names*}{xcolor}
%
\documentclass[10pt,]{krantz}
\usepackage{lmodern}
\usepackage{amssymb,amsmath}
\usepackage{ifxetex,ifluatex}
\usepackage{fixltx2e} % provides \textsubscript
\ifnum 0\ifxetex 1\fi\ifluatex 1\fi=0 % if pdftex
  \usepackage[T1]{fontenc}
  \usepackage[utf8]{inputenc}
  \usepackage{textcomp} % provides euro and other symbols
\else % if luatex or xelatex
  \usepackage{unicode-math}
  \defaultfontfeatures{Ligatures=TeX,Scale=MatchLowercase}
    \setmonofont[Mapping=tex-ansi,Scale=0.7]{Source Code Pro}
\fi
% use upquote if available, for straight quotes in verbatim environments
\IfFileExists{upquote.sty}{\usepackage{upquote}}{}
% use microtype if available
\IfFileExists{microtype.sty}{%
\usepackage[]{microtype}
\UseMicrotypeSet[protrusion]{basicmath} % disable protrusion for tt fonts
}{}
\IfFileExists{parskip.sty}{%
\usepackage{parskip}
}{% else
\setlength{\parindent}{0pt}
\setlength{\parskip}{6pt plus 2pt minus 1pt}
}
\usepackage{xcolor}
\usepackage{hyperref}
\hypersetup{
            pdftitle={Estadística},
            pdfauthor={Ricardo Michel MALLQUI BAÑOS},
            colorlinks=true,
            linkcolor=Maroon,
            filecolor=Maroon,
            citecolor=Blue,
            urlcolor=Blue,
            breaklinks=true}
\urlstyle{same}  % don't use monospace font for urls
\usepackage{longtable,booktabs}
% Fix footnotes in tables (requires footnote package)
\IfFileExists{footnote.sty}{\usepackage{footnote}\makesavenoteenv{longtable}}{}
\setlength{\emergencystretch}{3em}  % prevent overfull lines
\providecommand{\tightlist}{%
  \setlength{\itemsep}{0pt}\setlength{\parskip}{0pt}}
\setcounter{secnumdepth}{5}

% set default figure placement to htbp
\makeatletter
\def\fps@figure{htbp}
\makeatother

\usepackage[spanish,es-lcroman]{babel}
\usepackage{booktabs}
\usepackage{graphicx}
\usepackage{amsmath}
\usepackage{makeidx}
\makeindex
%\usepackage{showframe}
%\usepackage[a4paper]{geometry}
%\geometry{verbose,tmargin=3cm,bmargin=3cm,lmargin=3.5cm,rmargin=3cm}

\usepackage{times}
\renewcommand{\rmdefault}{ptm}
\usepackage[lite,subscriptcorrection,nofontinfo,zswash]{mtpro2}

\usepackage{graphicx}

% Determine if the image is too wide for the page.
\makeatletter
\def\ScaleIfNeeded{%
  \ifdim\Gin@nat@width>\linewidth
    \linewidth
  \else
    \Gin@nat@width
  \fi
}
\makeatother

% Resize figures that are too wide for the page.
\let\oldincludegraphics\includegraphics
\renewcommand\includegraphics[2][]{%
  \oldincludegraphics[scale=0.85]{#2}
}

\usepackage{amsthm}
\makeatletter
\def\thm@space@setup{%
  \thm@preskip=8pt plus 2pt minus 4pt
  \thm@postskip=\thm@preskip
}
\makeatother



\flushbottom 

\frontmatter
\usepackage[]{natbib}
\bibliographystyle{apalike}

\title{Estadística}
\author{Ricardo Michel MALLQUI BAÑOS}
\providecommand{\institute}[1]{}
\institute{Universidad Nacional San Cristóbal De Huamanga}
\date{2020-03-24}

\usepackage{amsthm}
\newtheorem{theorem}{Teorema}[chapter]
\newtheorem{lemma}{Lema}[chapter]
\newtheorem{corollary}{Corolario}[chapter]
\newtheorem{proposition}{Proposición}[chapter]
\newtheorem{conjecture}{Conjectura}[chapter]
\theoremstyle{definition}
\newtheorem{definition}{Definición}[chapter]
\theoremstyle{definition}
\newtheorem{example}{Ejemplo}[chapter]
\theoremstyle{definition}
\newtheorem{exercise}{Ejercicio}[chapter]
\theoremstyle{remark}
\newtheorem*{remark}{Observación}
\newtheorem*{solution}{Solución}
\let\BeginKnitrBlock\begin \let\EndKnitrBlock\end
\begin{document}
\maketitle

%\cleardoublepage\newpage\thispagestyle{empty}\null
%\cleardoublepage\newpage\thispagestyle{empty}\null
%\cleardoublepage\newpage
\thispagestyle{empty}
\begin{center}
\includegraphics{U.pdf}
\end{center}

%\setlength{\abovedisplayskip}{-5pt}
%\setlength{\abovedisplayshortskip}{-5pt}

{
\hypersetup{linkcolor=}
\setcounter{tocdepth}{2}
\tableofcontents
}
\listoftables
\listoffigures
\newcommand{\N}{\mathbb{N}}
\newcommand{\R}{\mathbb{R}}
\newcommand{\CC}{\mathbb{C}}
\newcommand{\I}{\mathbb{I}}
\newcommand{\f}{\mathbb{f}}
\newcommand{\X}{\mathbb{X}}
\newcommand{\D}{\mathbb{D}}
\newcommand{\Z}{\mathbb{Z}}
\newcommand{\Q}{\mathbb{Q}}
\newcommand{\norm}[1]{\left\Vert#1\right\Vert}
\newcommand{\abs}[1]{\left\vert#1\right\vert}
\newcommand{\set}[1]{\left\{#1\right\}}
\newcommand{\seq}[1]{\left<#1\right>}
\newcommand{\co}[1]{\left[#1\right]}
\newcommand{\cc}[1]{\left(#1\right)}
\newcommand{\J}{\mathcal{J}}
\newcommand{\K}{\mathcal{K}}
\newcommand{\M}{\mathcal{M}}
\newcommand{\F}{\mathcal{F}}

\hypertarget{resumen}{%
\chapter*{Resumen}\label{resumen}}


Este libro sobre la estadistica descriptiva. cuyo objetivo es demostrar resultados basicos muy útiles en el desarrollo de investigaciones.

\[\sum_1^2\]

\hypertarget{introducciuxf3n}{%
\chapter*{Introducción}\label{introducciuxf3n}}


\[\sum_1^2\]

\[\vec{u}=(1,1)-\rho\int_2^3\]

Debido a la poca información estructurada de estadistica descriptiva se propone escribir este libro con un enfoque demostrativo.

\[x^2+y^2\]

\mainmatter

\hypertarget{vectores}{%
\chapter{Vectores}\label{vectores}}

\hypertarget{intro}{%
\chapter{Rectas}\label{intro}}

\hypertarget{lugar-geomuxe9trico}{%
\chapter{Lugar geométrico}\label{lugar-geomuxe9trico}}

\[\frac{u}{\vec{u}}\]

\hypertarget{sed}{%
\section{sed}\label{sed}}

\[\frac{u}{\vec{u}}\]

\hypertarget{traductor}{%
\chapter{traductor}\label{traductor}}

\hypertarget{traducto2}{%
\chapter{traducto2}\label{traducto2}}

\hypertarget{estadistica}{%
\chapter{estadistica}\label{estadistica}}

\BeginKnitrBlock{theorem}
\protect\hypertarget{thm:unnamed-chunk-1}{}{\label{thm:unnamed-chunk-1} }En la elipse se verifican las siguientes igualdades

\begin{enumerate}
\def\labelenumi{\arabic{enumi}.}
\item
  \(d\left[B_1;F_i\right]=d\left[B_2;F_i\right]=a\)
\item
  \(d\left[V_1;C\right]=d\left[V_2;C\right]=a\)
\item
  \(d\left[C;\mathcal{L}_1\right]=d\left[C;\mathcal{L}_2\right]=\frac{c}{e}\)
\item
  \(c=d\left[P;F_1\right]=d\left[P;F_2\right]\) entonces \(c=ae\)
\end{enumerate}
\EndKnitrBlock{theorem}

\BeginKnitrBlock{proof}
\iffalse{} {Demostracion. } \fi{}
1. Ya que \(d\left[B_1;F_1\right]+d\left[B_1;F_2\right]=2a=d\left[B_2;F_1\right]+d\left[B_2;F_2\right]\) es decir \(2d\left[B_1;F_i\right]=2a=2d\left[B_2;F_i\right]\) entonces \(d\left[B_1;F_i\right]=a=d\left[B_2;F_i\right]\) \(i=1,2\).

\begin{enumerate}
\def\labelenumi{\arabic{enumi}.}
\setcounter{enumi}{1}
\item
  Por la definición \eqref{eq:binom} de la elipse se tiene
  \begin{equation}
  d\left[V_1;F_2\right]+d\left[V_1;F_1\right]=2a
  \label{eq:er}
  \end{equation}
  además la diferencia
  \begin{equation}
  d\left[V_1;F_2\right]-d\left[V_1;F_1\right]=2c
  \label{eq:err}
  \end{equation}
  restando las ecuaciones \eqref{eq:er} y \eqref{eq:err} se tiene
  \begin{equation}
  d\left[V_1;F_1\right]=a-c
  \label{eq:errr}
  \end{equation}
  entonces haciendo uso de \eqref{eq:errr} en \(d\left[V_1;C\right]=d\left[V_1;F_1\right]+d\left[F_1;C\right]=(a-c)+c=a\); de manera similar para el vértice \(V_2\).
\item
  En efecto \[\frac{d\left[B;F_i\right]}{d\left[B;\mathcal{L}_i\right]}=e\Longleftrightarrow \frac{a}{d\left[B;\mathcal{L}_i\right]}=e\] además \(d\left[B_i;\mathcal{L}_i\right]=d\left[C;\mathcal{L}_i\right]\) por lo tanto \(\frac{a}{d\left[C;\mathcal{L}_i\right]}=e\).
\item
  Pues \[\frac{d\left[P;F_1\right]}{d\left[P;\mathcal{L}_1\right]}=e\] implica \(\frac{a-c}{\frac{a}{e}-a}=e\) es decir \(c=ae\).
\end{enumerate}

Por lo tanto
\EndKnitrBlock{proof}

\hypertarget{eee}{%
\chapter{eee}\label{eee}}

Example text outside R code here; we know the value of
pi is In this section, we give a very brief introduction to Pandoc's Markdown. Readers who are familiar with Markdown can skip this section. The comprehensive syntax of Pandoc's Markdown can be found on the Pandoc website \url{http://pandoc.org}.

\begin{quote}
``I thoroughly disapprove of duels. If a man should challenge me,
I would take him kindly and forgivingly by the hand and lead him
to a quiet place and kill him.''
\end{quote}

In this section, we give a very brief introduction to Pandoc's Markdown. Readers who are familiar with Markdown can skip this section. The comprehensive syntax of Pandoc's Markdown can be found on the Pandoc website \url{http://pandoc.org}. \(\sum_1^2\)

\begin{quote}
I thoroughly disapprove of duels. If a man should challenge me,
I would take him kindly and forgivingly by the hand and lead him
to a quiet place and kill him.

-- Mark Twain
\end{quote}

\[\begin{pmatrix}\alpha & \beta\\
\gamma & \delta
\end{pmatrix}-\frac{2}{3} \begin{pmatrix}\alpha_1 & \beta_2\\
\gamma & \delta
\end{pmatrix}\]

\[X = \begin{bmatrix}1 & x_{1}\\
1 & x_{2}\\
1 & x_{3}
\end{bmatrix}\]

\[\begin{vmatrix}a & b\\
c & d
\end{vmatrix}=ad-bc\]

\[\begin{array}{ccc}
x_{11} & x_{12} & x_{13}\\
x_{21} & x_{22} & x_{23}
\end{array}\]

\hypertarget{appendix-apendice}{%
\appendix \addcontentsline{toc}{chapter}{\appendixname}}


\hypertarget{ecuaciones-de-primer-grado}{%
\chapter{Ecuaciones de primer grado}\label{ecuaciones-de-primer-grado}}

\[\int_1^3=\lim_{n\to \infty}\sum_{i=0}^{n}f^i(x)\]
citado por \citep{xie2015}

\hypertarget{raices-de-una-ecuacion-de-segundo-grado}{%
\section{Raices de una ecuacion de segundo grado}\label{raices-de-una-ecuacion-de-segundo-grado}}

\hypertarget{propiedades-de-una-ecuacion-de-segundo-grado}{%
\section{Propiedades de una ecuacion de segundo grado}\label{propiedades-de-una-ecuacion-de-segundo-grado}}

\hypertarget{ecacuaciones-lineales-de-primer-grado}{%
\chapter{Ecacuaciones lineales de primer grado}\label{ecacuaciones-lineales-de-primer-grado}}

\hypertarget{soluciones-de-ecuacuiones-lineales-de-primer-grado}{%
\section{Soluciones de ecuacuiones lineales de primer grado}\label{soluciones-de-ecuacuiones-lineales-de-primer-grado}}

\hypertarget{soluciones}{%
\section{Soluciones \ldots{}}\label{soluciones}}

\hypertarget{forma-matricial-de-una-ecuaciuxf3n-lineal}{%
\section{Forma matricial de una ecuación lineal}\label{forma-matricial-de-una-ecuaciuxf3n-lineal}}

\bibliography{book.bib,packages.bib}

\printindex

\end{document}
